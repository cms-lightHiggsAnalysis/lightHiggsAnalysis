\section{Event selection\label{sec:analysisregions}}
\Editor{Kyle Tos}

For the events passing the \texttt{HLT\_IsoMu24} or \texttt{HLT\_IsoTkMu24} trigger, 
the following selections are used to identify the important objects in the event
to optimize the sensitivity to our signal.
They include the dimuon and $\Pgtm\Pgth$ objects,
both of which come from the two light pseudoscalars, $\Pga$.
The selection below is defined as the \textit{preselection}.
The \textit{preselection} is applied to all samples in the four ABCD regions discussed in the Section~\ref{sec:fakes}. 

This analysis is designed to select the $\PH \rightarrow \Pgamm\Pgatt$ decay channel.
The selected events are required to pass:
\begin{itemize}
  \item Trigger: \verb|HLT_IsoMu24_v* OR HLT_IsoTkMu24_v*|
  \item Dimuon selection
  \begin{itemize}
    \item Two opposite-sign muons
    \item Leading isolated muon ($\Pgm_1$) $\pt > 26 \GeV$ and matched to the trigger
    \item Subleading muon ($\Pgm_2$) $\pt > 3 \GeV$
    \item $m(\Pgm_1, \Pgm_2) < 30 \GeV$
    \item $\DR(\Pgm_1, \Pgm_2) < 1.0$
  \end{itemize}
  \item Ditau selection
  \begin{itemize}
    \item Opposite-sign muon-tau pair
    \item Muon ($\Pgtm$) $\pt > 3 \GeV$
    \item Muon cleaned hadronic tau ($\Pgth$) $\pt > 10 \GeV$
    \item $\DR(\Pgtm, \Pgth) < 0.8$
  \end{itemize}
  \item Cross-cleaning
  \begin{itemize}
    \item $\DR(\Pgm_1, \Pgtm) > 0.4$
    \item $\DR(\Pgm_2, \Pgtm) > 0.4$
    \item $\DR(\Pgm_1, \Pgth) > 0.8$
    \item $\DR(\Pgm_2, \Pgth) > 0.8$
  \end{itemize}
  \item Veto event if \PGth jet passes CSVv2 Medium WP
\end{itemize}

From this data set, events are further categorized into four regions
base on $\Pgm_2$ and $\Pgth$ isolation.
\begin{itemize}
    \item Region A: isolated $\Pgm_2$, isolated $\Pgth$
    \item Region B: isolated $\Pgm_2$, anti-isolated $\Pgth$
    \item Region C: anti-isolated $\Pgm_2$, isolated $\Pgth$
    \item Region D: anti-isolated $\Pgm_2$, anti-isolated $\Pgth$
\end{itemize}
Region A will sometimes be referred to as the \textit{signal region}
and region B will sometimes be referred to as the \textit{sideband}.
These regions are used to estimate background contributions
in the signal region through a data driven fakerate method.
Regions C and D will be used to validate this method.
Figure~\ref{fig:ABCD} shows a graphic of the regions and their definitions.
This is described in more detail in Section~\ref{sec:fakes}.

\begin{figure}
  \centering
  \includegraphics[scale=.675]{figures/ABCD.png}
    \captionsetup{format=hang}
  \caption{Graphic depicting ABCD method.}
  \label{fig:ABCD}
\end{figure}

