\subsection{Application of scale factors\label{sec:controlregions}}

\Editor{Mengyao Shi}

To validate the scale factors applied, we use Drell--region which contains well understood Z handle to make sure all corrections on MC are being applied correctly. 
In addition, this control region is used to calculate the
misidentified object efficiency described in Section~\ref{sec:fakes}.

The selection sequence for the Drell--Yan control region is:
\begin{itemize}
    \item Trigger: \verb|HLT_IsoMu24_v* OR HLT_IsoTkMu24_v*|
    \item Muons must pass \verb|isMediumMuon| with ICHEP 2016 recommended modifications
    \item Select two opposite-sign isolated muons (delta beta corrected, $I_{rel} < 0.25$
    \item Leading muon ($\Pgm_1$) $\pt > 25 \GeV$
    \item Subleading muon ($\Pgm_2$) $\pt > 5 \GeV$
    \item $60 < m(\Pgm_1, \Pgm_2) < 120 \GeV$
\end{itemize}

In addition to the analysis selections listed above,
the standard assortment of MET filters are applied.
This includes the bad PF muon filter developed specifically
for the 2016 ReReco dataset.

In general, the most up-to-date muon POG group recommended
scale factors and corrections are applied to each event. 
\begin{itemize}
    \item Single muon trigger efficiency~\cite{muonpog:efficiencies}
    \item Muon identification and isolation efficiency
    \begin{itemize}
        \item POG values for $\pt > 20 \GeV$~\cite{muonpog:efficiencies}
        \item Calculated from $J/\Psi$ for identification with $\pt < 10 \GeV$ (Section~\ref{sec:tagandprobe})
    \end{itemize}
    %\item Tracking efficiency~\cite{muonpog:tracking}
    \item Rochester corrections for muon \pt~\cite{muonpog:rochester}
    \item Pileup reweighting~\cite{lumipog:pileup}
    \item Generator event weights
\end{itemize}

Our MC is uniformly higher than data, but shape agrees quite well. Therefore we applied a scale factor 0.92 on MC when making our final plots. 
The agreement between simulation and data after all corrections are applied
can be seen in Figures~\ref{fig:drellyanleps}-\ref{fig:drellyanz}.

% NOTE: this is all very standard and doesn't warrant detailed description unless we do something special
% if you really want to include it, a lot of language editing is necessary


%Muon identification scale factor: Remove fake muons is important for this control region study. 
%fake muons can occur for example in hadronic decays, or hadron shower remnant in
%the muon system that are reconstructed as a muon object.
%Muon identification relies on tracker and muon system information, such as the number of pixel hits, the number of tracker
%layers with hits, the muon segments reconstructed in the muon station, as well as quality
%requirements on the muon track, such as the track chi-square, the number of muon chamber
%hits in the track fit. However, muon identification has different efficiencies in data and MC,
%scale factor is introduced as ratio of efficiency of data versus efficiency of MC.
%This identification efficiencies was thoroughly studied by muon POG group on 2016 data,
%which has been valuable contribution to our control region study.
%Basically scale factors that is a function of pt, eta and LHC run era
%is applied to each MC event and scaled up or down to target a better match of data.
%
%Muon isolation scale factor: Besides muon identification,
%require muon to be isolated further more decrease chances that muon could be fake muons.
%And muon POG group kindly provides this efficiency discrepancy between 2016 data and MC
%for everyone who use muons in final states. Scale factors can be calculated by following equation.
%\begin{equation}
%SF_{ID and ISO}=SF_{ID/Tracker_track}*SF_{ISO/ID}
%\end{equation}
%
%In above equation, $SF_{ID and ISO}$ is scale factor for both,
%$SF_{ID/Tracker_track}$ is the identification SF, using as denominator muon tracker tracks; in the mean time, 
%$SF_{ISO/ID}$ the isolation SF, using as denominator muon tracker tracks with the corresponding ID
%selection applied.
%
%Similarly, muon tracking efficiency scale factor and muon trigger
%efficiency scale factor was also applied to incorporate MC/data's difference.
%
%Rochester correction: This correction was developed to remove a bias
%of the muon momentum from any detector misalignment or any possible error of the magnetic field.
%Therefore while looping over all events, we apply Rochester correction based on knowledge of eta, pt, phi of muon. 
%
%Pileup reweighting: Production Monte Carlo samples are generated with distributions
%for the number of pileup interactions. However this distribution does not exactly
%match the conditions for each data-taking period. The final distribution for the number
%of reconstructed primary vertices is sensitive to the details of the primary vertex
%reconstruction and to differences in data versus MC. Therefore this difference need
%to be taken care of. The approach is as following: Given a total inelastic cross section,
%the average bunch instantaneous luminosity can be directly converted into the expected number
%of interactions per crossing for this LumiSection. We used standard "pileupCalc" tool and by
%feeding into collision json file together with recommended minbias cross section,
%we could get an output of both MC and data. Taking the ratio of data and MC,
%we could get pileup reweighting scale factor, and apply to each individual Monte Carlo event.  
%
%generation level event weights:

\begin{figure}[htbp]
  \centering
  \subfloat[][$\PGm_{1}\ \pt$ ]{\includegraphics[width=0.4\textwidth]{figures/pt_of_Mu1.png}}
  \subfloat[][$\PGm_{2}\ \pt$ ]{\includegraphics[width=0.4\textwidth]{figures/pt_of_Mu2.png}} \\ 
  \subfloat[][$\PGm_{1}\ \eta$]{\includegraphics[width=0.4\textwidth]{figures/Eta_of_Mu1.png}}
  \subfloat[][$\PGm_{2}\ \eta$]{\includegraphics[width=0.4\textwidth]{figures/Eta_of_Mu2.png}}
  \caption{
           Drell--Yan control region lepton kinematic variables.
           }
  \label{fig:drellyanleps}
\end{figure}

\begin{figure}[htbp]
  \centering
  \subfloat[][$m_{\PGm\PGm}$   ]{\includegraphics[width=0.4\textwidth]{figures/invMass_of_Mu1_Mu2.png}}
  \subfloat[][$\DR(\PGm\PGm)$  ]{\includegraphics[width=0.4\textwidth]{figures/dRMu1Mu2Wider.png}} \\
  \subfloat[][$\PZ\ \pt$ ]{\includegraphics[width=0.4\textwidth]{figures/Mu1Mu2Pt.png}}
  \subfloat[][$\PZ\ \eta$]{\includegraphics[width=0.4\textwidth]{figures/Mu1Mu2Eta.png}}
  \caption{
           Drell--Yan control region di-muon kinematic variables.
           }
  \label{fig:drellyanz}
\end{figure}

